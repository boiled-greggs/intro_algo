%algo
\documentclass{article}
\usepackage[margin=1.0in]{geometry}
\usepackage{amsmath, amssymb, mathrsfs}
\usepackage[english]{babel}
\usepackage{graphicx}
\usepackage{enumerate}
\usepackage{listings}

\title{Intro to Algorithms HW 2}
\author{Greg Stewart}
\date{\today}

\begin{document}

\maketitle

\section*{Q1} Compute $2^{2^n}$ in linear time.

\begin{lstlisting}
function power_of_power_of_two(n):
  input: the power n to raise the 2 in the exponent to
  output: the answer

  result = 2
  for i in 0..n:
    result = result * result
  
  return result

\end{lstlisting}

Since we assume multiplication of arbitrary size integers takes unit time, the above algorithm is $O(n)$.

\smallskip

If we don't make that assumption, then every multiplication in the for loop takes $O(m^2)$ time, where $m$

is the number of bits. The numbers being multiplied are $n$-digits, though, so overall the complexity 

becomes $O(n) \cdot O(n^2) = O(n^3)$.

\section*{Q2} 

\subsection*{(a) \normalsize $N$ is an $n$-bit number. How many bits is $N!$?}

$ $

Multiplication of 2 $m$-bit numbers results in a number of $2m$ bits. Given a number $N$, the number of 

bits is $n = log_2{N}$. With a factorial, the number $N$ is multiplied with all numbers before it:

$$N! = N \cdot (N-1) \cdot (N-2) \cdots 1$$

Since multiplication roughly results in adding the bit-lengths of the numbers being multiplied, we can 

write the resulting bit-length of $N!$ as

$$m = \log(N+1) + \log(N) + \log(N-1) + \cdots + \log(1)$$

$N$ could be any arbitrary number, so what this amounts to is on the order of $$m = \log(N)^N$$

Which we can rewrite as $$m = N\log(N)$$

Recalling that the number of bits in $N$ is $n$, we finally arrive at the answer

$$m = N \cdot n$$

This is linear, of course, so the resulting $O()$ notation is $$O(n)$$.

\subsection*{(b) \normalsize Give an algorithm to compute $N!$ and analyze the running time.}

\begin{lstlisting}
function factorial(n):
  if n = 1 or 0: return 1
  answer = 1
  for i in 1..n+1:
    answer = answer*i
  return answer
\end{lstlisting}

For the trivial case, this is $O(1)$. In general, however, this requires $n$ multiplications, where $n$ is the 

number to be factorialized. The bit complexity of multiplication is $O(n^2)$, and this operation occurs $n$ 

times, so we end up with $$O(n^3)$$

\bigskip

\section*{Q3 \normalsize Find the GCD of 1492 and 1776 using}

\subsection*{(a) \normalsize the prime factorization method and using Euclid's method}

\textit{Euclid's Method:} We use the extended method for simplifying part (b)

\begin{table}[h!]
  \centering
  \begin{tabular} {c c | c c}
    $a$ & $b$ & $r = a\mod b$ & combination \\ [0.5ex]
    \hline
    1776 & 1492 & 284 & 1776 - 1492 \\
    1492 & 284 & 72 & 1492 - 5(284) \\
    284 & 72 & 68 & 284 - 3(72) \\
    72 & 68 & 4 & 72 - 68 \\
    68 & 4 & 0 & - \\
  \end{tabular}
\end{table}

So gcd(1776, 1492) = 4.

\noindent\textit{Prime Factorization:} Factor both and obtain the common prime factor.

\begin{align*}
  1492 &= 2(746) = 2^2 (373) \\
  1776 &= 2^4 (111) = 2^4 \cdot 3 (37)
\end{align*}

From the factorization it is obvious that the only common factor is $2^2$, so gcd = 4,

\subsection*{(b) \normalsize express the GCD as an integer linear combination of two inputs}

To do this, we work backwards from the GCD, using the table from Euclid's Method.

\begin{align*}
  4 &= 72 - 68 \\
  &= (1492 - 5(284)) - (284 - 3(72)) \\
  &= (1492 - 5(1776 - 1492)) - ((1776-1492) - 3(1492 - 5(284))) \\
  &= (1492 - 5(1776 - 1492)) - ((1776-1492) - 3(1492 - 5(1776 - 1492)))\\
  &= 1492 - 5(1776) + 5(1492) - (1776 - 1492 - 3(1492) + 15(1776 - 1492) \\
  &= 6(1492) - 5(1776) - 1776 + 1492 + 3(1492) - 15(1776) + 15(1492) \\
  &= 25(1492) - 21(1776)
\end{align*}

With the final expression being the linear combination of the inputs.








\end{document}
