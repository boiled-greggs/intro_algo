\documentclass[12pt]{article}
\usepackage[margin=1.0in]{geometry}
\usepackage{amsmath, amssymb, mathrsfs}

\title{Intro to Algorithms HW 1}
\author{Greg Stewart}
\date{\today}

% problems: DPV 0.3; q2 - q4 on web page
\begin{document}

\maketitle

\section*{Q1}

Fibonacci numbers are given by $$F_0 = 0 \text{, } F_1 = 1 \text{, } F_n = F_{n-1} + F_{n-2} $$

(a) \textit{Problem.} Prove $F_n \geq 2^{0.5n}$ for $n \geq 6$.

\smallskip

\textbf{Proof.} For n = 6 (the base case), we have $$F_6 = 8 \text{ and } 2^{0.5*6} = 2^3 = 8$$ $$F_6 \geq 2^{0.5*6} = 8$$

So the base case is true. Let's assume that this is also true for some $k > 6$, i.e. $$F_k = F_{k-1} + F_{k-2} \geq 2^{0.5k}$$ 

and 
$$F_{k-1} \geq 2^{0.5(k-1)}$$

Then for the $k+1$ term, we have $$F_{k+1} = F_k + F_{k-1}$$

So
$$F_{k+1} \geq 2^{0.5k} + 2^{0.5(k-1)}$$
$$F_{k+1} \geq  2^{0.5(k-1)} (2^{0.5} + 1)$$

Since $2^{0.5} + 1 > 2$, we can write that
$$F_{k+1} \geq 2 \cdot 2^{0.5(k-1)} = 2^{0.5k + 0.5}$$

So we see that
$$F_{k+1} \geq 2^{0.5(k+1)}$$

\smallskip
%%%%%%%%%%%%%%%%%%%%%%%%%%%%%%%%%%%%%%%

%%%%%%%%%%%%%%%%%%%%%%%%%%%%%%%%%%%%%%%%%%%
(c) \textit{Problem.} Largest $c$ for which $F_n = \Omega(2^{cn})$

\smallskip

We know that $$F_n = F_{n-1} + F_{n-2}$$

so we can rewrite the inequality as $$F_n \geq 2^{c(n-1)} + 2^{c(n-2)}$$ $$F_n \geq 2^{cn} (2^{-c} + 2^{-2c} )$$

and for the statement to remain true, we need $2^{-c} + 2^{-2c} = 1$.

$$2^{-c} + 2^{-2c} = 1$$
$$2^{-2c} + 2^{-c} - 1 = 0$$

Solving this like a quadratic shows that

$$2^{-c} = -\frac{1 \pm \sqrt{5}}{2}$$
$$c = -\log_2(-\frac{1 \pm \sqrt{5}}{2})$$

Only one of these values is positive, so we must choose that one:

$$c = 0.69424$$


%%%%%%%%%%%%%%%%%%%%%%%%%%%%%%%%

\bigskip

\section*{Q2}

Let's take a look at some cases:

\begin{itemize}
\item $n = 0 \rightarrow$ 1 * printed
\item $n = 1 \rightarrow$ 2 * printed
\item $n = 2 \rightarrow$ 1 * + 2 * + 1 * = 4 * printed
\item $n = 3 \rightarrow$ 1 * + 1 * + 2 * + 4 * = 8 * printed
\end{itemize}

So we have $$T(0) = 1, \text{ } T(1) = 2, \text{ }  T(2) = 4, \text{ }  T(3) = 8, \dots$$

This pattern clearly shows exponential appreciation, and can be written as 

$$T(n) = T(n-1) + T(n-2) + T(n-3) + \cdots + T(0)$$

and we know from this that $$T(n-1) = T(n-2) + T(n-3) + \cdots + T(0)$$

So we can write a formula for $T(n)$:

$$T(n) = 2 T(n-1)$$

This formula is equivalent to 

$$T(n) = 2^n$$

To show this we use induction. $$T(0) = 2^0 = 1$$

Assume that $T(k) = 2^k$ for some $k > 0$. Then

$$T(k+1) = 2^{k+1} = 2^{k} \cdot 2^1 = 2 \cdot 2^k = 2 T(k)$$

Which is the same as the formula that we found at first. So it's clear that

$$T(n) = 2^n$$



%n = 0 --> 1 || n=1 --> 2 || n=2 --> 4 || n=3 --> 8

\bigskip
%%%%%%%%%%%%%%%%%%%%%%%%%%%%%%%%%%%%%%%
\section*{Q3}
$$$$

(a) $f = n(n+1) \text{ and } g = 2000n^2$ $$f = \Theta(g)$$

(b) $f = 100n^2 \text{ and } g = 0.01n^3 $ $$f = O(g)$$

(c) $f = \log_2 n \text{ and } g = \ln n $ $$f = O(g)$$

(d) $f = \log_2^2 n \text{ and } g = \log_2 n^2 $ $$f = \Omega(g)$$

(e) $f =  2^{n-1} \text{ and } g = 2^n$ $$f = \Theta(g)$$

(f) $f =  (n-1)! \text{ and } g = n! $ $$f = O(g)$$


\bigskip
%%%%%%%%%%%%%%%%%%%%%%%%%%%%%%%%%%%%%%%%
\section*{Q4}

In order of growth from lowest to highest:

$$ 5 \log_2(n+100)^{10} \text{, } \ln^2 n \text{, }  n^{1/3} \text{, } 0.001n^4 + 3n^3 + 1 \text{, } 2^{2n} \text{, } 3^n \text{, } (n-2)! $$










\end{document}