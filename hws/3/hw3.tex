\documentclass{article}
\usepackage[margin=1.0in]{geometry}
\usepackage{amsmath, amssymb, mathrsfs}
\usepackage[english]{babel}
\usepackage{graphicx}
\usepackage{enumerate}
\usepackage{listings}
\usepackage{tikz}

\title{Intro to Algorithms HW 3}
\author{Greg Stewart}
\date{\today}

\begin{document}

\maketitle

\section*{Q1}

\textit{Problem.} Show whether $4^{1536} \equiv 9^{4824} \mod 35$

Well first off, for any prime $p$, we know $x^{p-1} \equiv 1 \mod p$. 35 is not prime, but its factorization is $5 \cdot 7$, and we can split up the statement so $$x^{5-1} \equiv 1 \mod 5 \text{ and } x^{7-1} \equiv 1 \mod 7$$ From this we can see that $$(x^{5-1})^{7-1} = x^{24} \equiv 1 \mod (5\cdot 7) = 1 \mod 35$$ for all $x$ such that $1 \leq x < 35$. Now all we need to do is factor the exponents a bit. $$4^{1536} = 4^{24 \cdot 64} \equiv 1 \mod 35$$ and $$9^{4824} = 9 ^{24 \cdot 201} \equiv 1 \mod 35$$ So finally we can see that $$4^{1536} \equiv 9^{4824} \mod 35$$


\section*{Q2}

\textit{Problem.} Solve $x^{86} \equiv 6 \mod 29$

From Fermat's Little Theorem, $$x^{28} \equiv 1 \mod 29$$ so we have $$x^{86} \equiv x^{2} \mod 29$$ and using the original problem, $$x^2 \equiv 6 \mod 29$$ which is fortunately similar to $$x^2 \equiv 64 \mod 29$$ and means that we can write $$x^2 - 64 \equiv 0 \mod 29$$ $$(x-8)(x+8) = 0 \mod 29$$ which has solutions $x = 8$ and $x = 21$ so $$x \equiv 8, 21 \mod 29$$ 

\section*{Q3}

Prove that $\gcd(F_{n+1}, F_n) = 1$ for $n \geq 1$ where $F_n$ is the $n$-th Fibonacci element.

The $n=1$ case is clearly true, as $\gcd (1,1) = 1$. Now suppose that the statement is true for some $n \geq 1$, so $\gcd (f_{n+1}, f_n) = 1$. Then for $n+1$ we have

$$\gcd (f_{n+2}, f_{n+1}) = \gcd (f_{n+1}, f_{n+2} - f_{n+1})$$ 

which, by the definition of a Fibonacci number, is just $$\gcd (F_{n+1}, F_n) = 1$$

So we have shown that $\gcd(F_{n+1}, F_n) = 1$.


\section*{Q4}

Multiplying n-bit by an m-bit integer is $O(nm)$. Given x and y, give an efficient algorithm to compute $x^y$.

The algorithm is as follows:

\begin{lstlisting}
function power(x, y):
  input: base x and exponent y
  output: x^y

  if y = 0:
    return 1
  if y = 1:
    return x
  if y is even:
    return power(x*x, y/2) 
  else
    return x*power(x*x, (n-1)/2)

\end{lstlisting}

The problem of $x^y$ can be broken down into products of $x^2$, essentially. For example, $x^8 = (x^2)^{8/2} = ((x^2)^2)^2$. In general, if $y$ is even, then

$$x^y = (x^2)^{y/2}$$

and if $y$ is odd, then

$$x^y = x(x^2)^{(n-1)/2}$$

Unless $y =0$ or $y=1$, this is exactly what the above algorithm does, recursively dividing $y$ by 2 for each call. So this is a sort of divide-and-conquer algorithm. There are $\log y$ squarings of $x$ and in the worst case $\log y$ multiplications. In this case $y$ is an $m$-bit number, so $\log y = m$. Squaring the $n$-bit $x$ is $O(n^2)$, done $m$ times, so we're at $O(mn^2)$, and since there are $m$ multiplications we finally end up with $$O((mn)^2)$$ 






















\end{document}
